
%%%%%%%%%%%%%%%%%%%%%%%%%%%%%%%%%%%%%%%%%%%%%%%%%%%%%%%%%%%%%%%%%%%%%%%%%%%%%%%%%%%%%%%%%%%%%%%%%
%%
%%  Die vorliegenden LaTeX-Folien stehen Mitarbeiter*innen und Studierenden der Universität Wien
%%  zur Verfügung und sind ausschließlich zur Verwendung in Forschung und Lehre der Universität  
%%  Wien vorgesehen. Das Copyright der LaTeX-Vorlagen liegt bei der Universität Wien.
%%
%%  These LaTeX slides are available to employees and students of the University of Vienna 
%%  and are intended exclusively for use in research and teaching at the University of Vienna. 
%%  The copyright of the LaTeX templates is held by the University of Vienna.
%%
%%%%%%%%%%%%%%%%%%%%%%%%%%%%%%%%%%%%%%%%%%%%%%%%%%%%%%%%%%%%%%%%%%%%%%%%%%%%%%%%%%%%%%%%%%%%%%%%%


\documentclass[hyperref={pdfpagelabels=false}, aspectratio=169, t]{beamer}  %% Choose aspectratio=43 or aspectratio=169

	
%%%%%%%%%%%%%%%%%%%%%%%%%%%%%%%%%
%% ====== Define Style ======= %% 
%%%%%%%%%%%%%%%%%%%%%%%%%%%%%%%%%

%% ======== required inputs ========

	%% ====== title page ======
	\title{Focus Project: \\ On Pseudo-Random Number Generators}											%% Presentation Title
	\newcommand{\titleBackground}{0}    						%% Background graphic: 0 = no; 1 = yes; 2 = yes with more text space
	\newcommand{\gPath}{Graphics/}									%% set graphics path
	\newcommand{\graphicsTitleBackground}{Lombardi2014.png}   %% Filename of background graphic


%% ======== optional inputs ========
	
	%% ====== title page ======
	\subtitle{Simone Spedicato, BSc \\ 2025S 280522-1 Methods of Computational Astrophysics, 30/06/2025}                   					%% (optional) Subtitle, comment out to avoid include
	% \newcommand{\authorText}{Simone Spedicato, B.Sc.}				%% (optional) Author
	\newcommand{\logoTitleFooterR}{logo_astro.jpg}  %% (optional) additional logo in title page footer, most right
	%\newcommand{\logoTitleFooterM}{grey}  %% (optional) additional logo in title page footer, more right (if horizontal spacing not adequate, save multiple logos in one file, use \logoTitleFooterR)
	%\newcommand{\logoTitleFooterL}{grey}  %% (optional) additional logo in title page footer, right (if horizontal spacing not adequate, save multiple logos in one file, use \logoTitleFooterR)

	%% ====== footer ======
	% \newcommand{\textFooter}{Footer text (author, presentation title, version, etc.)} %% (optional) Text for footline, e.g. title, comment out lines to avoid include, max. 1 line
	\newcommand{\slideNumberLabelFooter}{Slide}    	%% Page/Slide/Folie/Seite
	\date{30 June 2025}              									%% Date, comment out to get current date

	%% ====== header ====== 
	%\newcommand{\sectionHeader}{Section (optional)} %% (optional) Text before section number, e.g. Section or Kapitel; comment out to avoid headline

	%% ====== TOC ====== 
	%\newcommand{\includeTocAtBeginSection}{}   		 %% (optional) Table of contents at begin of every section, comment out to avoid include


%% ======== load beamer style ========
	\usepackage{beamerthemeuniwien2017}
	
	
%%%%%%%%%%%%%%%%%%%%%%%%%%%%%%%%%%%%%
%% ====== Further Preamble ======= %% 
%%%%%%%%%%%%%%%%%%%%%%%%%%%%%%%%%%%%%

%% ====== settings (optional) ======
	\usenavigationsymbolstemplate{}     %% (optional) Comment out to include navigation 


%%%%%%%%%%%%%%%%%%%%%%%%%%%%%%%%%%%
%% ====== Begin Document ======= %% 
%%%%%%%%%%%%%%%%%%%%%%%%%%%%%%%%%%%

\begin{document}

%% ====== Title Page ======

\maketitle
											
													
%% ====== Outline at beginning of document ======

%\begin{frame}{Outline (optional)}
%	\tableofcontents							%% (optional) Table of contents, comment out lines to avoid include
%\end{frame}


%% ====== Actual Slides  ======

% Slide 2: introduction -- setting the stage, big picture

\begin{graphicsFrame}{Introduction}{short}{0.62}{left}{random_walk}{Random walk in 2D (Wikimedia Commons, CC BY-SA 4.0)}
	\textbf{Goals:}
	\begin{itemize}
		\item Explore how psuedo-random number generators (PRNGs) work.
		\item Test the limitations of various PRNGs. 
		\item Apply PRNGs to the modeling of a star cluster.
	\end{itemize}
	\vspace{0.8em}
	\textbf{Methods:}
	\begin{itemize}
		\item Literature review on PRNGs.
		\item Implement PRNGs and test suites.
		\item Simulate a star cluster using PRNGs.
	\end{itemize}
\end{graphicsFrame}

\begin{graphicsFrame}{Outlook}{short}{0.60}{left}{simulations}{Preliminary test results for the LCG.}
	\textbf{Done:}
	\begin{itemize}
		\item LCGs, Mersenne Twister, Common Entropy Sources, RANDU.
		\item Basic test suite.
	\end{itemize}
	\vspace{0.8em}
	\textbf{Next Steps:}
	\begin{itemize}
		\item More test scenarios.
		\item Comparison with simulations.
		\item Write the report.
	\end{itemize}
\end{graphicsFrame}


%% ====== Slides Templates ======		

% \begin{graphicsFrame}{Layout ``Body with figure, small right''}{short}{0.7}{left}{graphic_rs}{\textcopyright~Universität Wien/derknopfdruecker.com}

% 		Random formula
% 		\[
% 			(0,1)\ni t\mapsto\frac{\partial}{\partial t} g(t,\omega)=\int_{( 0,1-t]}\frac{G(dr,\omega)}{1-r}
% 		\]
% 		Another random formula
% 		\begin{equation}\label{eq1}
% 			\int_{( G(0+,\cdot),1)}\frac{ f_{\mathcal{G},G^{\leftarrow}(t,\cdot),X}}{1-G^{\leftarrow}(t,\cdot)}\,dt
% 			= f_{\mathcal{G},G,X}\quad \textrm{a.s.}
% 		\end{equation}
% 		And another, even more random formula
% 		\[
% 			\mathbb{P}(X\leq Z-\varepsilon)\leq
% 			\mathbb{P}(X\leq q_{\mathcal{G},\delta}(X)-\varepsilon )< \delta
% 		\]

% \end{graphicsFrame}									

% \begin{textFrame}{Layout ``Titel und Inhalt'' = Standardlayout Überschriften}{}{Referenz, Quellen- oder Copyright-Angabe bei Bedarf einfügen}

% 	\begin{itemize}
% 		\item Fließtext 22 pt, Mustertext: Weit hinten, hinter den Wortbergen, fern der Länder Vokalien und Konsonantien leben die Blindtexte.
% 		\item Abgeschieden wohnen sie in Buchstabhausen an der Küste des Semantik, eines großen Sprachozeans. Ein kleines Bächlein namens Duden fließt durch ihren Ort und versorgt sie mit den nötigen Regelialien.
% 		\item Es ist ein paradiesmatisches Land, in dem einem gebratene Satzteile in den Mund fliegen. Nicht einmal von der allmächtigen Interpunktion werden die Blindtexte beherrscht – ein geradezu unorthographisches Leben.

% 	\end{itemize}
% \end{textFrame}

% \begin{textFrame}{Layout ``Titel und Inhalt wenig Text''}{0.7}{Referenz, Quellen- oder Copyright-Angabe bei Bedarf einfügen}

% 	\begin{itemize}
% 		\item Mustertext: Weit hinten, hinter den Wortbergen, fern der Länder Vokalien und Konsonantien leben die Blindtexte.
% 		\item Abgeschieden wohnen sie in Buchstabhausen an der Küste des Semantik, eines großen Sprachozeans. Ein kleines Bächlein namens Duden fließt durch ihren Ort und versorgt sie mit den nötigen Regelialien.
% 		\item Nicht einmal von der allmächtigen Interpunktion werden die Blindtexte beherrscht – ein geradezu unorthographisches Leben.
% 	\end{itemize}
% \end{textFrame}


% %% ====== Section  ======

% \section{Layout ``Abschnittsüberschrift''}

% \begin{sectionFrame}{section169.jpg}{mit Untertitel}
% \end{sectionFrame}


% \section{Layout ``Abschnittsüberschrift ohne Bild''~-- Titel kann auch mehrzeilig sein}

% \begin{sectionFrame}{}{mit Untertitel}
% \end{sectionFrame}


% \begin{textFrame2}{Layout ``Zwei Inhalte''}{}{
% 		\begin{itemize}
% 			\item In die Inhaltsplatzhalter können unterschiedliche Elemente eingefügt werden.
% 			\item Mustertext: Abgeschieden wohnen sie in Buchstabhausen an der Küste des Semantik, eines großen Sprachozeans. 
% 					Ein kleines Bächlein namens Duden fließt durch ihren Ort und versorgt sie mit den nötigen Regelialien.
% 		\end{itemize}
% }{Referenz, Quellen- oder Copyright-Angabe}{}{\includegraphics[width=.7\linewidth]{\gPath diagram.jpg}}%
% {Referenz, Quellen- oder Copyright-Angabe}
% \end{textFrame2}

% \begin{textFrame2}{Layout ``Vergleich''}{Vorteile des XYZ-Modells in Zusammenhang \newline mit dem Projekt}%
% {
% 		\begin{itemize}
% 			\item Weit hinten, hinter den Wortbergen, fern der Länder Vokalien und Konsonantien leben die Blindtexte.
% 			\item Abgeschieden wohnen sie in Buchstabhausen an der Küste des Semantik, eines großen Sprachozeans. 
% 		\end{itemize}
% }{}{Nachteile des XYZ-Modells in Zusammenhang \newline mit dem Projekt}{
% 		\begin{itemize}
% 			\item Abgeschieden wohnen sie in Buchstabhausen an der Küste des Semantik, eines großen Sprachozeans.
% 			\item Ein kleines Bächlein namens Duden fließt durch ihren Ort und versorgt sie mit den nötigen Regelialien. 
% 		\end{itemize}
% }{}
% \end{textFrame2}

% \begin{graphicsFrame}{Layout ``Inhalt mit Bild \\ klein rechts''}{short}{0.7}{left}{graphic_rs}{\textcopyright~Universität Wien/derknopfdruecker.com}

% 		\begin{itemize}
% 			\item Abgeschieden wohnen sie in Buchstab-hausen an der Küste des Semantik, eines großen Sprachozeans.
% 			\item Ein kleines Bächlein namens Duden fließt durch ihren Ort und versorgt sie mit den nötigen Regelialien.
% 			\item Es ist ein paradiesmatisches Land, in dem einem gebratene Satzteile in den Mund fliegen. Abgeschieden wohnen sie in Buchstabhausen an der Küste des Semantik, eines großen Sprachozeans. 
% 		\end{itemize}

% \end{graphicsFrame}

% \begin{graphicsFrame}{Layout ``Inhalt mit Bild klein \\ links''}{short}{0.7}{right}{graphic_ls}{\textcopyright~Universität Wien/derknopfdruecker.com}

% 		\begin{itemize}
% 			\item Abgeschieden wohnen sie in Buchstab-hausen an der Küste des Semantik, eines großen Sprachozeans.
% 			\item Ein kleines Bächlein namens Duden fließt durch ihren Ort und versorgt sie mit den nötigen Regelialien.
% 			\item Es ist ein paradiesmatisches Land, in dem einem gebratene Satzteile in den Mund fliegen. Abgeschieden wohnen sie in Buchstabhausen an der Küste des Semantik, eines großen Sprachozeans.
% 		\end{itemize}

% \end{graphicsFrame}

% \begin{graphicsFrame}{Layout ``Inhalt mit Bild größer rechts''}{}{0.47}{left}{graphic_rm}{\textcopyright~Universität Wien/Barbara Mair}

% 		\begin{itemize}
% 			\item Abgeschieden wohnen sie in Buchstab-hausen an der Küste des Semantik, eines großen Sprachozeans.
% 			\item Ein kleines Bächlein namens Duden fließt durch ihren Ort und versorgt sie mit den nötigen Regelialien.
% 		\end{itemize}

% \end{graphicsFrame}

% \begin{graphicsFrame}{Layout ``Inhalt mit Bild größer links''}{}{0.54}{right}{graphic_lm}{\textcopyright~Universität Wien/Barbara Mair}

% 		\begin{itemize}
% 			\item Abgeschieden wohnen sie in Buchstab-hausen an der Küste des Semantik, eines großen Sprachozeans.
% 			\item Ein kleines Bächlein namens Duden fließt durch ihren Ort und versorgt sie mit den nötigen Regelialien.
% 		\end{itemize}

% \end{graphicsFrame}

% \begin{graphicsFrame}{Layout ``Bild groß mit Titel''}{}{0.1}{}{graphic_xl}{\textcopyright~Universität Wien/Barbara Mair}

% \end{graphicsFrame}

% \begin{graphicsFrame}{}{}{0.43}{right}{graphic_ll}{\textcopyright~Universität Wien/derknopfdruecker.com}

% 		\begin{itemize}
% 			\item Layout „Bild groß mit Text“ 
% 			\item Abgeschieden woh-nen sie in Buchstab-hausen an der Küste des Semantik, eines großen Sprachozeans.
% 			\item Ein kleines Bächlein namens Duden fließt durch ihren Ort und versorgt sie mit den nötigen Regelialien.
% 			\item Es ist ein paradiesma-tisches Land, in dem einem gebratene Satzteile in den Mund fliegen.
% 		\end{itemize}

% \end{graphicsFrame}

% \begin{graphicsFrame}{Layout ``Bild abfallend mit Titel''}{}{-1}{}{graphic_xxl}{\textcopyright~Universität Wien/derknopfdruecker.com}

% \end{graphicsFrame}

% \begin{graphicsFrame2}{}{0.39}{graphic_2c_lm}{\textcopyright~Universität Wien/derknopfdruecker.com}{graphic_2c_rm}{\textcopyright~Universität Wien/Barbara Mair}

% 		\begin{itemize}
% 			\item Layout „Twei Bilder mit Text rechts“ 
% 			\item Ein kleines Bächlein namens Duden fließt durch ihren Ort und versorgt sie mit den nötigen Regelialien.
% 			\item Es ist ein para-diesmatisches Land, in dem einem gebratene Satzteile in den Mund fliegen.
% 		\end{itemize}

% \end{graphicsFrame2}

% \begin{graphicsFrame2}{Layout ``Titel, zwei Bilder mit Text rechts''}{0.39}{graphic_2c_ls}{\textcopyright~Universität Wien/Barbara Mair}{graphic_2c_rs}{\textcopyright~Universität Wien/Barbara Mair}

% 	\textbf{Ein kleines Bächlein namens Duden} fließt durch ihren Ort und versorgt sie mit den nötigen Regelialien.
% 	\smallskip

% 	Es ist ein paradiesmatisches Land, in dem einem gebratene Satzteile in den Mund fliegen.

% \end{graphicsFrame2}

% \begin{graphicsFrame2}{Layout ``Titel, zwei Bilder''}{0}{graphic_2c_ll}{\textcopyright~Universität Wien/Barbara Mair}{graphic_2c_rl}{\textcopyright~Universität Wien/derknopfdruecker.com}

% \end{graphicsFrame2}


%% ====== End Document ====== %%
\end{document}